\documentclass[a4paper]{IEEEtran}

\usepackage[utf8]{inputenc}
\usepackage{graphicx}
\usepackage{amsmath}
\usepackage{amssymb}
\usepackage{mathtools}
\usepackage{subcaption}
\PassOptionsToPackage{hyphens}{url}\usepackage{hyperref}
\usepackage{textgreek}
\usepackage{pgf}
\usepackage{import}
\usepackage{float}

\mathtoolsset{showonlyrefs}

\markboth{Seminar WS 20/21: Anthropomatik: Von der Theorie zur Anwendung}{Seminar WS 20/21: Anthropomatik: Von der Theorie zur Anwendung}

\title{Overview of Variational Autoencoder-based Generative Models}

\author{Maximilian~Schik}

\begin{document}

\maketitle

\begin{abstract}
This work tries to give the reader an overview over generative models based on the variational autoencoder framework and how they learn to disentangle the latent features of given data. It explains the basic functionality of autoencoders and how different models improved upon that. One of those models is the Variational Autoencoder. It introduces the concept of the Evidence Lower Bound and utilizes it to learn the latent features of a given probability distribution. The $\beta$-Variational Autoencoder is one improvement of this concept that performs better at disentangling such learnt features. This aspect is further improved upon by the $\beta$-Total Correlation Autoencoder. It outperforms $\beta$-Variational Autoencoder at the disentanglement of the latent features.
\end{abstract}

\section{Introduction}
A generative model deals with the modeling of probability distributions from raw data \cite{doersch2016tutorial}. After training, this model can be used for the generation of new data instances. Having accurate models of random processes is advantageous for many fields. It could, for example, be used to enhance a dataset with more samples or for detecting outliers or anomalies. When generating samples, those with high probability are desired to be more likely than those with lower probability \cite{doersch2016tutorial}. At the same time, at least a low amount of control about the samples being generated is beneficial. For example, when generating 3d-models of trees for a video game \cite{doersch2016tutorial}, it might be convenient to choose the "style" of tree (conifer or deciduous) or the color of the leaves. Learning an efficient representation of such higher dimensional features is one goal of generative models. In the most ideal situation, those features should also be disentangled. Take the task of generating human faces as an example: When adjusting the hair style of a generated face, the gender should not change.

While those models are autoassociativ, which means that they try to map an input to itself, only the decoder part of the models is used as a generator. This is done by sampling noise from a probability distribution and using it as an input for the decoder. The architecture of an \textit{Autoencoder} (\textbf{AE}) was first proposed in \cite{ballard1987modulalearning}. There it is defined as an autoassociative model trained by mean square error. Its intended use was to extract features from input data for use in different machine learning algorithms as a way to deal with the low performing hardware used at that time \cite{ballard1987modulalearning}. Today it is used for many more applications, like noise reduction or anomaly detection. AEs are not really capable of generating meaningful data, but they are the basis for the other models discussed here.

An enhanced version of the AE is the \textit{Vartiational Autoencoder} (\textbf{VAE}) as proposed in \cite{kingma2014autoencoding}. It has a similar architecture to an AE, but the VAE differs in the loss function and the way its latent variables are learned. The \textit{Beta Variational Autoencoder} (\textbf{\textbeta -VAE}) is a further improvement of the VAE proposed by \cite{higgins2017vae}. It introduces a regularization of the latent space and improves the disentanglement of the latent features. \textit{Beta Total Correlation Variational Autoencoders} (\textbf{\textbeta -TCVAE}) \cite{chen2019isolating} are another improvement made by decomposing the \textit{Evidence Lower Bound} (\textbf{ELBO}) used by VAEs and \textbeta -VAEs and then analyzing it using concepts taken from Information Theory. It even further improves the disentanglement of \textbeta -VAE and learns an even more efficient representation of inputs in its latent space.

In the next four sections each model will be discussed and the way it learns to represent the given data will be explained. It will also be discussed how each model improves over the previous. After that all models will be compared with each other. The report will end with a conclusion about the featured models.

\section{Autoencoder}
\subsection{Method}
AEs are the simplest of the discussed models. They consist of two parts: the encoder and the decoder. They will be treated as mathematical functions with $f : X \mapsto Z$ as the encoder and $g : Z \mapsto X$ as the decoder. $X$ will be called the data space and $Z$ the latent space. $f$ and $g$ will be learned from data. When implementing an AE, neural networks are used as function approximaters for $f$ and $g$.

The objective of AEs is the minimization of the loss function:

\begin{equation}
	L(\theta_1, \theta_2) := \frac{1}{n} \sum_{i = 1}^{n}{(x_i - g_{\theta_2}(f_{\theta_1}(x_i)))^2},
	\label{mse}
\end{equation}

where $\theta_1$ and $\theta_2$ are the parameters of $f$ and $g$, $n$ is the number of datapoints and $x_i$ is the $i$th datapoint. This loss function is also called \textit{mean square error} (\textbf{MSE}). During training, $f$ will learn an encoding of $x \in X$ in the latent space $f(x) \in Z$. The decoder will learn to decode a latent vector $z \in Z$ to a vector in the data space $g(z) \in X$. So, a perfect AE would map an input to itself, meaning $x = g(f(x))$. The MSE calculates the difference between the reconstruction $g(f(x))$ and the actual input $x$.

To prevent either $f$ or $g$ from just learning the identity function $id(x) = x$, an artificial information bottleneck is introduced by making $Z$ lower dimensional than $X$. Therefore, $f$ must learn a representation (encoding) of $x$ in its latent space. This is also called representation learning \cite{DBLP:journals/corr/abs-1206-5538}. 

\subsection{Latent Space}
For an efficient encoding, the latent space must have specific qualities: \textit{a)} latent vectors close to each other in $Z$ should produce samples close to each other in $X$ (\textbf{continuity}) and \textit{b)} every vector $z \in Z$ should produce meaningful data (\textbf{completeness}). Using MNIST digits as an example, two fours that look similar to each other should produce $z_1, z_2 \in Z$ that are close to each other. At the same time, a random sample $z \in Z$ should produce a meaningful digit \cite{burgess2018understanding}.

May $Z = \mathbb{R}^{2}$ be taken as an example. When encoding a digit $x \in X$ with $f(x) = z$, it is desired that the features of $x$ be represented by $z$. For example, the value of the digit ($n \in \{0,1,...,9\}$) could be one encoded feature, and the angle of the number another encoded feature. For an efficient encoding, each feature would be encoded in an own dimension, so they could be changed independently of each other.

\subsection{Problems with AE}
AEs are not used for generative purposes. The lack of structure in their latent space is the reason for that. The training of an AE doesn not necessarily produce a continuous and complete latent space. This leads to close inputs not being close in latent space, and sampled inputs not resulting in any meaningful outputs.

\section{Variational Autoencoder}
An enhancement that builds up upon the traditional AE is the Variational Autoencoder. It is a directed probabilistic graphical model \cite{kingma2014autoencoding}. It improves upon normal AEs by regulating the latent space and consequently motivates the model to learn a more efficient representation of inputs in the latent space.

\subsection{Encoding in VAE}
The assumption a VAE makes is that the data points $x \in X$ are produced by a random process, where first a $z$ is sampled from a $p(z)$, and then a $x$ is sampled from $p(x|z)$. Now, when defining the encoder from a probabilistic point of view, we want it to learn the distribution $p(z|x)$. Similarly, the decoder should learn the distribution $p(x|z)$. Sampling $z \sim p(z|x)$ would be equivalent to encoding an $x$ to $z$. The same applies for the decoder and $p(x|z)$ \cite{doersch2016tutorial}.

May $q(z|x)$ be an approximation of $p(z|x)$. The 
\textit{Kullback-Leibler divergence} (\textbf{KLD}) $\mathcal{D}(q \ \| \ p) \geq 0$ is a metric for comparing two probability distributions $q$ and $p$. $\mathcal{D}(q \ || \ p) = 0$ would be equivalent to $q = p$. To learn an accurate approximation of $p(z|x)$, the KLD
\begin{equation}
	\mathcal{D}\left(q(z|x) \ \| \ p(z|x) \right) = \mathbb{E}_{z \sim q}\left[ \log q(z|x) - \log p(z|x) \right]
	\label{EBOL1}
\end{equation}
has to be minimized. Minimizing \eqref{EBOL1} would result in a better approximation $q(z|x)$ of $p(z|x)$, and therefore, a better encoding of $x$ in $Z$. Rearranging with Bayes rule results in
\begin{equation}
	= \mathbb{E}_{z \sim q} \left[ \log q(z|x) - \log p(x,z) \right] + \log p(x)\,,
	\label{EBOL2}
\end{equation}

where the first term is the \textit{Evidence Lower Bound} ({\textbf{ELBO})
\begin{equation}
	ELBO = \mathbb{E}_{z \sim q} \left[ \log q(z|x) - \log p(x,z) \right]\,,
	\label{ELBO3}
\end{equation} 

with $p(x,z) = p(x|z)p(z)$. The ELBO is a lower bound for the log probability $\log p(x) \geq ELBO$ \cite{kingma_2019}. Plugging this into \eqref{EBOL2} allows to rewrite the equation as
\begin{equation}
	\log p(x) = ELBO - \mathcal{D}\left(q(z|x) \ \| \ p(z|x) \right)\,,
	\label{ELBO4}
\end{equation}

showing that minimizing $\mathcal{D}\left(q(z|x) \ \| \ p(z|x) \right)$ is equivalent to maximizing the ELBO, following from $\log p(x) \geq ELBO$ and $\mathcal{D}(q \ \| \ p) \geq 0$.

Using this knowledge \eqref{ELBO3} can be rearranged:
\begin{equation}
	ELBO = \mathbb{E}_{z \sim q}\left[ \log p(x|z) + \log p(z) - \log q(z|x) \right]
	\label{ELBO5}
\end{equation}
\begin{equation}
	= \mathbb{E}_{z \sim q}\left[ \log p(x|z) \right] - \mathcal{D}(q(z|x) \ \| \ p(z))\,.
	\label{ELBO6}
\end{equation}

The first term in \eqref{ELBO6} performs the same role as MSE for the traditional AE. It tries to maximize the probability of a latent vector $z \in Z$ sampled from $q(z|x)$ to be decoded into a $\tilde{x}$ being similar to $x$. As $p(x|z)$ will be modeled by the decoder $g : Z \mapsto X$, the negative MSE can be used as an approximation:
\begin{equation}
	ELBO = \mathbb{E}_{z \sim q}\left[ -||x - g(z)||^2 \right] - \mathcal{D}(q(z|x) \ \| \ p(z))\,.
	\label{ELBO7}
\end{equation}

As the real distribution of $p(z)$ is not known, it will be approximated by a normal distribution $p(z) \sim \mathcal{N}(0,1)$. $q(z|x)$ will be modeled by $\mathcal{N}(\mu(x), \Sigma(x))$, where $\mu(x)$ and $\Sigma(x)$ are two functions mapping an input $x \in X$ to a mean and a covariance in $Z$. These functions will be implemented as function approximators in the shape of neural networks and will be learned from data.

Plugging in the approximations gives the objective that the model has to maximize:
\begin{equation}
	\mathbb{E}_{z \sim q}\left[ -||x - g(z)||^2 \right] - \mathcal{D}(\mathcal{N}(\mu(x), \Sigma(x)) \ \| \ \mathcal{N}(0, 1))\,.
	\label{ELBO8}
\end{equation}

Because $z$ is still a random variable that gets sampled from a distribution and sampling is not continuous, gradient descent is not applicable to this objective. This can be taken care of by moving the sampling to the input using the \textit{reparameterization trick}: A random variable $\epsilon$ will be sampled from a normal distribution $\mathcal{N}(0,1)$ and transformed by $h(x, \epsilon) = \mu(x) + \sqrt{\Sigma(x)} \cdot \epsilon$ \cite{doersch2016tutorial}. 

Following this the objective can be rewritten as a loss function that the model has to minimize:
\begin{gather}
	\mathcal{L}(\theta_1, \theta_2, \theta_3) = \\ 
	||x - g_{\theta_3}(h(x, \epsilon))||^2 + \mathcal{D}(\mathcal{N}(\mu_{\theta_1}(x), \Sigma_{\theta_2}(x)) \ \| \ \mathcal{N}(0, 1))\,,
	\label{vae_loss}
\end{gather}
with $\epsilon \sim \mathcal{N}(0,1)$ and $x \in X$. $\theta_1, \theta_2$ and $\theta_3$ are the parameters or weights of the functions $\mu, \Sigma$ and $g$. The KLD of two normal distributions is trivial to compute.

\subsection{Improvements of VAE}
The VAE improves upon the traditional AE primarily in two points:
By encoding the input not in a single vector but into a probability distribution, an area of the latent space gets reserved for latent vectors that produce similar $x$ when decoded by the decoder. This addresses the problem of the continuous latent space. At the same time, by having a random aspect in the sampling of $z$, the decoder learns to deal with the "blurriness" of the latent vector. This helps with the completeness of the latent space, resulting in random samples $z \in Z$ producing more meaningful data. This makes the VAE capable of being used as a generator for meaningful data. This is done by detaching the decoder from the encoder and applying noise sampled from a distribution. When doing this, the encoder is not needed anymore and can be discarded.

\subsection{Problems with VAE}
While VAEs allow for the generation of meaningful data, they are still lacking structure in the latent space. Features tend to be entangled with each other, showing that the model does not learn the most efficient representation in the latent space. The optimization of the KLD also comes with its drawbacks in the shape of lower reconstruction accuracy. 

\section{Beta-Variational Autoencoder}
The $\beta$-VAE \cite{higgins2017vae} improves the disentanglement of VAEs by adding a single scalar to the objective:
\begin{equation}
	\mathbb{E}_{z \sim q}\left[ \log p(x|z) \right] - \beta \mathcal{D}(q(z|x) \ \| \ p(z)).
\end{equation}
If $\beta = 1$, the model is equivalent to a VAE. Higher values of $\beta$ enforce stricter regularizations on the latent space and improve the structure in the latent space. 

\subsection{Disentangled Feature Learning}
For $\beta > 1$, the KLD is weighted more in the overall objective, resulting in higher pressure on the model to minimize $\mathcal{D}(q(z|x) \ \| \ p(z))$. This means that the model gets pressured to learn $q(z|x)$ close to $p(z)$. Because we approximate $q(z|x)$ with $\mathcal{N}(\mu(x), \Sigma(x))$ and $p(z)$ with $\mathcal{N}(0, 1)$, this means that $\mu(x)$ will get closer to $0$, and $\Sigma(x)$ will get closer to $1$. Therefore, the encodings in the latent space will get squeezed together. A consequence of that is the overlapping of the distributions $\mathcal{N}(\mu(x), \Sigma(x))$. This would result in a higher reconstruction error $\mathbb{E}_{z \sim q}\left[ \log p(x|z) \right]$, as the sampling of $z ~ q(z|x_1)$, due to its random nature, could result in a $z$ that has a much higher probability under a different $x_2$. Therefore the decoder would map it with higher probability to $x_2$ rather than to $x_1$, even though it was sampled from $q(z|x_1)$ \cite{burgess2018understanding}. To reduce the reconstruction error again, the model has to encode $x_1, x_2 \in X$, that are close to each other in data space, close to each other in latent space, therefore reducing the amount of reconstruction error produced by the overlapping. Therefore, inputs that share features or are similiar to each other will be encoded close to each other. Inputs that differ only in one feature should be very close to each other, while inputs that share almost no features will be further apart, therefore separating and disentangling the features in the latent space \cite{burgess2018understanding}.


\subsection{Improvements of $\beta$-VAE}
Because $\beta$-VAE is equivalent to normal VAE for $\beta = 1$, $\beta$-VAE has no disadvantages over VAE. Experiments show \textbeta -VAEs performing better at disentangling latent features than normal VAEs. This comes from the more efficient use of the latent space motivated by the higher penality for the KLD introduced by \textbeta, however, it may sometimes be at a lower reconstruction accuracy. This leads to the reconstructed inputs looking more similiar to each other than the original inputs did. Higher values of $\beta$ result in lower reconstruction accuracy. Therefore a balance between reconstruction accuracy and disentanglement has to be found by adjusting $\beta$ accordingly. See some more detailed experimental results in \hyperref[sec:experiments]{Section VI}.


\section{Beta-Total Correlation Variational Autoencoder}
A further refinement of the VAE framework has been introduced by \cite{chen2019isolating} in the shape of the Beta Total Correlation Autoencoder. It improves the disentanglement even further by decomposing the ELBO and scaling the resulting terms.

\begin{figure*}[h]
	\centering
    \import{plots/disentanglement_comparison/}{disentanglement_comparison.pgf}
    \caption{Comparison of some of the disentangled features learned by the discussed models. Each row corresponds to one feature and every column to one model. The numbers in the brackets represent the range that the latent dimensions traverses from left to right. For comparison, the dimensions resembling the features the most are picked by hand for each model individually.}
    \label{fig:disentanglement}
\end{figure*}

\subsection{Decomposition and Analysis of the ELBO}
The KLD in the ELBO as seen in \eqref{ELBO6} can be expanded to
\begin{align*}
\mathcal{D}(q(z|x) \ \| \ p(z)) =
	\mathbb{E}_{z \sim q} \left[ \log q(z|x) - \log p(z) \right. \\
	 + \log q(z) - \log q(z) + \log \prod_j q(z_j) - \log \prod_j q(z_j) ]\,,
\end{align*}

which can be rearranged to 
\begin{align*}
	= \mathcal{D}(q(z, x) \ \| \ q(z)p(x)) \\ + \mathcal{D}(q(z) \ \| \  \prod_{j} q(z_j)) \\ + \sum_j \mathcal{D}(q(z_j) \ \| \ p(z_j))
\end{align*}
\begin{align*}
	= I[z;x] + \mathcal{D}(q(z) \ \| \  \prod_{j} q(z_j)) + \sum_j \mathcal{D}(q(z_j) \ \| \ p(z_j)) \,,
\end{align*}

where the first term will be called the \textit{index-code mutual information} (\textbf{MI}), the second term is known as the \textit{total correlation} (\textbf{TC}), and the third term will be called the \textit{dimension wise KLD} \cite{chen2019isolating}.

The mutual information $I\left[ z; x \right]$ is defined as
\begin{equation}
	I[z;x] = H(z) - H(z|x) = \mathcal{D}(q(z, x) \ \| \ q(z)p(x))\,,
\end{equation}
where $H(x)$ is the \textit{entropy} of a random variable. The entropy is a measurement of uncertainty about the random variable. So, the mutual information $I[z;x]$ is the amount the uncertainty about $z$ gets reduced by when given an observation about $x$. If the mutual information is zero, the variables are completely independent of each other.

The total correlation is a generalization of the mutual information for more than one variable. It measures the dependency among a set of random variables, in this case $z_1 \ldots z_n$. If the total correlation would be zero, it would mean that all $z_j$ are independent of each other and are therefore completely disentangled. As shown in \cite{chen2019isolating}, this term is the main reason why \textbeta -VAE is succesful at disentangling the latent features.

The dimension-wise KLD mainly acts as a restriction on the individual latent dimensions to keep them from deviating too far from the approximated normal distribution for $p(z) \sim \mathcal{N}(0,1)$.

An improved objective can be obtained by adding weights to the components:
\begin{align*}
	ELBO = \mathbb{E}_{z \sim q}\left[ \log p(x|z) \right] - \alpha I[z;x] \\ - \beta \mathcal{D}(q(z) \ \| \  \prod_{j} q(z_j)) \\ - \gamma \sum_j \mathcal{D}(q(z_j) \ \| \ p(z_j))\,.
\end{align*}

Experiments have shown that $\alpha = \gamma = 1$ and $\beta > 1$ gave the best results \cite{chen2019isolating}. With $\alpha = \gamma = 1$ the ELBO can be simplified to
\begin{align*}
	ELBO = \mathbb{E}_{z \sim q}\left[ \log p(x|z) \right]\\
	 - \mathcal{D}(q(z|x) \| p(z))\\
	 - (\beta - 1) \cdot \mathcal{D}(q(z) \| \prod_j q(z_j))
\end{align*}

\subsection{Improvements of $\beta$-TCVAE}
As is the same with $\beta$-VAE, for $\beta = 1$ $\beta$-TCVAE is equivalent to a normal VAE. The only noteworthy disadvantage therefore is the slightly more complex objective. In most other aspects $\beta$-TCVAE behaves like $\beta$-VAE. Because of that, it shares the same disadvantages of lower reconstruction accuracy for higher $\beta$ as $\beta$-VAE. Experiments have shown, as will be seen in \hyperref[sec:experiments]{Section VI}, that $\beta$-TCVAE outperforms the other models at disentangling the latent features. This is the result of a more specialized ELBO, which leads to a better organized latent space. 

\section{Experiments}
\label{sec:experiments}


For this report, one of each model has been trained to visualize the topics discussed in the previous sections. The code used for training the models and generating the figures is available on GitHub\footnote{\url{https://github.com/taDachs/Overview-of-Variational-Autoencoder-based-Generative-Models.git}}

\subsection{Experiment setting}
The training set used is the celeb\_a dataset \cite{liu2015faceattributes}. It contains over 200,000 different images of faces. To keep the model focused on face attributes the images have been cropped to contain as little background as possible. For the encoder a stack of six convolutional layers with leaky relu and batch normalization layers inbetween is used. The decoder is the reverse, with convolutional transpose instead of convolutional layers. For the probabilisic models (VAE, $\beta$-VAE, $\beta$-TCVAE) the encoder ends in two seperated dense layers that learn $\mu(x)$ and $\Sigma(x)$ as seen in \eqref{ELBO8}. The output of the decoder is used as the mean of a normal distribution, similiar to $\mu(x)$, with a fixed variance. The AE doesn't use any probability distributions and is fully deterministic.

The implementation is written in Python using Keras/Tensorflow. For modeling the normal distributions, tensorflow-probability layers are used, as they allow for easy sampling and probing of a paramterized distribution. Internally they use the reparameterization trick for sampling. Because $\Sigma(x)$ learns the variance of a normal distribution, which has to be $\geq 0$, the exponential function $\exp(x) = e^x$ is used on the output of $\Sigma(x)$.

Every model is trained for 30 epochs with a learning rate of 0.0001 and a batch size of 128 using ADAM\cite{DBLP:journals/corr/KingmaB14}. The dimension of the latent space is 32. $\beta$-VAE uses $\beta = 5$, while $\beta$-TCVAE uses $\beta = 20$. The negative loglikelihood is used as a loss function for the probabilistic models while MSE is used for the AE.

The samples are produced by sampling a latent vector from the prior $\mathcal{N}(0,1)$ and using it as an input for the decoder.

For $\mathbb{E}_{z \sim q}\left[ q(z) \right]$, the estimator shown in \cite{chen2019isolating} is used:
\begin{equation}
	\mathbb{E}_{z \sim q} \left[ q(z) \right] \approx \frac{1}{M} \sum_{i = 1}^{M} \left[ \log \frac{1}{NM} \sum_{j = 1}^{M}q(z_i|n_j) \right]\,.
\end{equation}

\subsection{Results}
The main two qualities by that the models should be evaluated are the generative abilites of the model and how many disentangled latent features it learns.

AE didn't generate any meaningful samples. The faces are distorted and are barely recognizable. The probabilistic models on the other hand all produced meaningful and recognizable data. This can be seen in \ref{fig:generative}. The samples are blurrier compared to other generative models like GANs \cite{goodfellow2014generative}. $\beta$-VAE and $\beta$-TCVAE seem to generate a more varied spectrum of samples compared to VAE.

\begin{figure}
	\centering
    \import{plots/generative_comparison/}{generative_comparison.pgf}
    \caption{Comparison of samples generated by each model. AE performs worse compared to \ref{fig:disentanglement}, because the latent vectors used here are fully sampled from the prior $p(z)$, while the ones used in \ref{fig:disentanglement} are sampled from $p(z|x)$.}
    \label{fig:generative}
\end{figure}

\begin{figure}
	\centering
    \import{plots/reconstruction_comparison/}{reconstruction_comparison.pgf}
    \caption{Comparison of different $\beta$ values for a $\beta$-TCVAE. For higher values of $\beta$, the reconstruction looses finer features like wrinkels or details of the hairstyle.}
    \label{fig:reconstruction}
\end{figure}

Regarding feature disentanglement, a better performance can be seen at higher regularizations of the latent space. While AEs output gets partly distortet and hard to recognize while traversing the latent dimension, the probabilistc models keep their general structure. Better regularization results in less features being entangled in each other. $\beta$-TCVAE outperforms the other models in that regard, being able to change the baldness or the existence of sunglasses while keeping most significant features of the face intact. This can be seen in \ref{fig:disentanglement}.

Increasing $\beta$ also makes the faces loose smaller or unique features. For example, clear glasses seem to get lost in the reconstruction and are not learned by the model. This may lead to faces looking more similiar to each other. This can be seen in \ref{fig:reconstruction}.

\subsection{Discussion}
The results of the experiments show that a higher and, more importantly, a specialized regularization of the latent space result in better disentanglement of the latent features. This can be seen when comparing $\beta$-VAE and $\beta$-TCVAE with each other. $\beta$-TCVAE performs better because it optimizes only the term that regularizes the independence of the latent dimensions, therefore reducing entanglement between dimensions.

The reduction in reconstruction accuracy can also been seen in the bluriness of the images. As mentioned before, when increasing $\beta$, a tradeoff has to be made between bluriness and disentanglement. Another tradeoff that should be considered is the dimension of the latent space. A too big latent space with too little regularization can lead to high entanglement, while a too small latent space with too much regularization can lead to the loss of features and therefore a lower reconstruction accuracy. A small latent space, combined with a high pressure for disentanglement, means less space for learned features. As high regularization pressures the model to encode every feature in its own latent dimension, more features have to be ignored and are therefore lost in the reconstruction. Too little regularization on the other side can result in more entanglement, because the model tries the optimize $\log p(x|z)$ by storing more information in the latent space. Bigger latent spaces on the other hand can lead to bigger models, resulting in longer training times.

Related work like \cite{DBLP:journals/corr/BaoCWLH17} has shown that the issue of bluriness can be addressed by combining VAEs and GANs. This gives a model that is able to generate realistic looking images while still giving control over the features of the generated image.

\section{Conclusion}
In this report, various VAE-based models have been discussed. Deterministic AEs lack the ability to generate data, due to their complete lack of regularization in the latent space. VAE uses the KLD to introduce such a regularization, resulting in a latent space that can be used for the generation of new data. By scaling the KLD, $\beta$-VAE further improves this, pressuring the model to use its latent space more efficient and therefore disentangling the latent features. The decomposition of the regularization into its components and scaling those more specialized terms lets $\beta$-TCVAE use its latent space even more efficient, allowing it to disentangle the latent features better than the other models.

Future work could focus one the evaluation of the disentanglement. One method already proposed uses trained linear classifiers \cite{higgins2017vae}. Another method uses a metric based on \textit{mutual information} \cite{chen2019isolating}. With a better disentanglement metric than visual inspection by a person, it would be easier to compare different models and hyperparameters with each other.

\bibliographystyle{unsrt}
\bibliography{references}
\end{document}
